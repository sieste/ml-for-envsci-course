\documentclass{beamer}

\usetheme{Warsaw}
\author{Stefan Siegert}
\date{}

\setbeamertemplate{page number in head/foot}[framenumber]
\setbeamertemplate{navigation symbols}{}
%\beamertemplatenavigationsymbolsempty

\AtBeginSection[]
{%
    \begin{frame}
        \tableofcontents[currentsection]
    \end{frame}
}%

% \AtBeginSubsection[]
% {%
%     \begin{frame}
%         \tableofcontents[currentsection,currentsubsection]
%     \end{frame}
% }%


% https://tex.stackexchange.com/a/69721/82534 "latex beamer toc"
%\setbeamertemplate{sections/subsections in toc}[]
%\setbeamertemplate{section in toc}[ball unnumbered]
\setbeamertemplate{section in toc}{\textbf{\inserttocsection}}

\usepackage{listings}
\usepackage{xcolor}
\usepackage{bm}
\usepackage{tikz}



\newcommand{\fat}[1]{\boldsymbol{#1}}

\definecolor{codegreen}{rgb}{0,0.6,0}
\definecolor{codegray}{rgb}{0.5,0.5,0.5}
\definecolor{codepurple}{rgb}{0.58,0,0.82}
\definecolor{backcolour}{rgb}{0.95,0.95,0.92}

\lstdefinestyle{mystyle}{
    backgroundcolor=\color{backcolour},    
    commentstyle=\color{codegreen},
    keywordstyle=\color{magenta},
    %numberstyle=\tiny\color{codegray},
    stringstyle=\color{codepurple},
    basicstyle=\ttfamily\tiny,
    breakatwhitespace=false,         
    breaklines=true,                 
    captionpos=b,                    
    keepspaces=true,                 
    %numbers=left,                    
    %numbersep=5pt,                  
    showspaces=false,                
    showstringspaces=false,
    showtabs=false,                  
    tabsize=2
}

\lstset{style=mystyle}





\title{Introduction to Machine Learning for Environmental Science}


\begin{document}

\maketitle

\begin{frame}
This 2-day workshop introduces core concepts of machine learning
including common algorithms, standard workflows and key software
libraries. The focus will be on supervised machine learning for
regression and classification using a range of methods such as random
forests and deep convolutional neural networks. The material will be
taught through a mix of theory and practical examples across different
environmental sciences such as meteorology and ecology.
\end{frame}

\section{Introduction}

\subsection{Supervised ML}



\begin{frame}{Supervised machine learning}
\begin{itemize}
\item We usually have an target (outcome) measurement that we wish to predict. 
\item The prediction should be based on a set of features (inputs). 
\item For example
\begin{itemize}
\item predict future temperature based on current temperature, pressure and wind speed 
\item predict occurrence of wildfire based on measures of heat, drought and wind 
\item predict crop yield based on precipitation, temperature, soil nutrients, pests.
\end{itemize}
\item The goal of supervised ML is to build a model that predicts outcomes based on available features in new, previously unseen situations.
\item A "good" model is one that produces accurate predictions.
\end{itemize}
\end{frame}


\begin{frame}{Regression and classification}
\begin{itemize}
\item The nature of the target variable determines the type of ML problem.
\item Regression: Predicting quantitative outcomes. 
\begin{itemize}
\item For example: temperature in degrees, crop yield in tonnes, number of sharks, precipitation amount in mm
\end{itemize}
\item Classification: Predicting categorical outcomes. 
\begin{itemize}
\item For example: wildfire occurrence, temperature below zero, precipitation type (rain/snow/hail), precipitation amount none/low/medium/high
\end{itemize}
\end{itemize}
\end{frame}


\subsection{Model training}


\begin{frame}{Function approximation}
\begin{itemize}
\item A ML model is a prediction rule that translates one or more features \(x_1, x_2, \dots\) into an estimate \(\hat{y}\) of the target \(y\).
\item The prediction rule is usually calculated by a specific mathematical function \[\hat{y} = f(x_1, x_2, \dots)\]
\item The function \(f(\dots)\) usually includes one or many parameters \(\theta_1, \theta_2, \dots\) that control how the function translates inputs to outputs.
\item Defining the feature vector \(\fat{x} = (x_1, x_2, \dots)\) and parameter vector \(\fat{\theta}=(\theta_1, \theta_2, \dots)\) a ML model is often compactly written as \[\hat{y} = f(\fat{x}; \fat\theta)\]
\end{itemize}
\end{frame}

\begin{frame}{Training data}
\begin{itemize}
\item Supervised ML works by "learning from examples".
\item Training data set is a set of input-output examples \[S = \left\{(\fat{x}_i, y_i)\right\}_{i=1}^n\] where
\begin{itemize}
\item \(\fat{x}_i = (x_{i,1}, x_{i,2}, \dots, x_{i,k})\) is the feature vector (containing \(k\) features) in the \(i\)-th example
\item \(y_i\) is the outcome in the \(i\)-th example
\item \(n\) is the sample size
\end{itemize}
\item The training data set \(S\) is used to build our prediction model to predict a new outcome \(y^*\) based on a feature vector \(\fat{x}^*\).
\end{itemize}
\end{frame}


\begin{frame}{Empirical loss minimisation}
\begin{itemize}
\item Model training is achieved by 
\begin{itemize}
\item trying to find model parameters \(\fat\theta\) such that
\item the model outputs \(\hat{y}_1, ..., \hat{y}_n\) calculated from \(\fat{x}_1, \dots, \fat{x}_n\) 
\item are as "close" as possible to the training targets \(y_1, \dots, y_n\)
\item where closeness is measured by the loss function \(L(\hat{y}, y)\)
\end{itemize}
\[\hat{\fat\theta} = \underset{\fat\theta}{\mathrm{argmin}}\sum_{i=1}^n L(f(\fat{x}_i; \fat\theta), y_i)\]
\item The trained model can then be used to make predictions \(\hat{y}^*\) for new outcomed \(y^*\) based on new feature vectors \(\fat{x}^*\)
\[\hat{y}^* = f(\fat{x}^*; \hat{\fat\theta})\]
\end{itemize}
\end{frame}

\begin{frame}{In summary ...}
To train a ML model by empirical loss minimisation we need:
\begin{itemize}
\item a training data set \(S\) containing \(n\) examples of inputs and corresponding outcomes
\item a model function with trainable parameters \(\fat\theta\) that translates inputs \(\fat{x}\) into an output \(\hat{y}\)
\item a loss function that quantifies how close model output \(\hat{y}\) is to the target outcome \(y\)
\item a mechanism to find parameters that minimise the loss 
\end{itemize}
\end{frame}

\subsection{Examples, model evaluation}


\begin{frame}{Example: Linear regression}
\begin{itemize}
\item Training features \(x_1, \dots, x_n\) are temperatures at different times \(t_1, \dots, t_n\)
\item Training targets \(y_1, \dots, y_n\) are temperatures a short time \(\Delta\) later \(t_1 + \Delta, \dots, t_n + \Delta\)
\item Model function is a linear function with intercept \(\theta_1\) and slope \(\theta_2\), such that \[\hat{y} = f(x; \fat\theta) = \theta_1 + \theta_2 \cdot x\]
\item Our loss function is the squared-error loss \[L(\hat{y}, y) = (\hat{y} - y)^2\]
\item Optimisation: exaustive search over a grid of parameter values (don't do this in practice!)
\end{itemize}
\end{frame}

\begin{frame}{Example: Linear regression}
\begin{center}
\includegraphics[width=.28\textwidth]{fig/linreg-xy.png}
\includegraphics[width=.3\textwidth]{fig/linreg-xyplot1.png}
\end{center}

\begin{center}
\includegraphics[width=.37\textwidth]{fig/linreg-thetaheatmap.png}
\includegraphics[width=.35\textwidth]{fig/linreg-xyplot2.png}
\includegraphics[width=.25\textwidth]{fig/linreg-xyyhat.png}
\end{center}
\end{frame}

\begin{frame}[fragile]{Machine Learning in python with \texttt{scikit-learn}}
\begin{center}
\texttt{https://scikit-learn.org}
\includegraphics[width=\textwidth]{fig/sklearn.png}
\end{center}
\end{frame}


\begin{frame}[fragile]{Linear Regression in Scikit-Learn}
\begin{lstlisting}[language=python]
import numpy as np
from sklearn.linear_model import LinearRegression

# load file
t7110 = np.loadtxt("t7110.dat", comments="#")

# extract features (july temperatures) and targets (august temperatures)
x = t7110[:, 7]
y = t7110[:, 8]

# sklearn expects k features to be stored in a 2d array with k columns
x = x.reshape(-1, 1)   

# initialise a linear regression model
model = LinearRegression()

# fit the model to data
model.fit(x, y)

# extract parameter estimates from fitted model
theta_hat = np.array([model.intercept_, model.coef_[0] ])

# calculate predictions in training data
y_hat = model.predict(x)

# calculate predictions on new data
x_new = np.linspace(15, 20, 100).reshape(-1,1)
y_new = model.predict(x_new)
\end{lstlisting}
\end{frame}



\begin{frame}{How good is my model?}
\begin{itemize}
\item The model outputs \(\hat{y}_1, \dots, \hat{y}_n\) be compared to the targets \(y_1, \dots, y_n\) using a suitable error metric.
\item For example root mean squared error (RMSE) \[RMSE = \sqrt{\frac1n \sum_{i=1}^n (\hat{y}_i - y_i)^2}\]
\item Our model has \(RMSE \approx 0.9\) -- Is that "good"?
\item Getting \(RMSE > 0\) means our model is not perfect, but how bad is it?
\item To judge whether the model is useful we should compare it to a suitable benchmark. 
\end{itemize}
\end{frame}


\begin{frame}{Benchmarking}
\begin{itemize}
\item A benchmark model (or reference model) for performance evaluation is usually one or several of
\begin{itemize}
\item An alternative (competing) model for the same target.
\item Climatology (constant mean): \[\hat{y}^{(clim)}_i = \frac1n \sum_{i=1}^n y_i\]
\item Persistence (last available observation of the target); here: \[\hat{y}^{(pers)}_i = x_i\]
\item Any other simple estimate of the target that could be calculated with reasonable effort given the same inputs as our model.
\item In fact, linear regression itself is often a good reference to benchmark more complicated ML model.
\end{itemize}
\end{itemize}
\end{frame}

\begin{frame}{Skill scores}
\begin{itemize}
\item Given a suitable error metric, we calculate
\begin{itemize}
\item \(S\): the error of our model (averaged over a test data set)
\item \(S_{ref}\): the mean error of our chosen reference model 
\item \(S_{perf}\): the mean error or a hypothetical perfect model that outputs \(\hat{y}_i = y_i\) each time; (usually \(S_{perf} = 0\))
\end{itemize}
\item The skill score of our model relative to the benchmark is then defined as
\[Skill = \frac{S_{ref} - S}{S_{ref} - S_{perf}}\]
\item \(Skill \le 0\): Our model is not better than the benchmark.
\item \(Skill \in (0, 1)\): Our model improves over the benchmark. 
\item \(Skill = 1\): Our model is perfect.
\end{itemize}
\end{frame}





\begin{frame}{Example: k-nearest-neighbor classification}
\begin{itemize}
\item given an input \(\fat{x}^*\) find the \(k\) closest features \(\fat{x}^{(1)}, \dots, \fat{x}^{(k)}\) in the training data, and the corresponding outcomes \(y^{(1)}, \dots, y^{(k)}\)
\item "closeness" between the feature vectors \(\fat{x}^{(i)}\) and \(\fat{x}^{(j)}\) is defined in terms of the Euclidean distance \(\sqrt{\sum_m (x^{(i)}_m - x^{(j)}_m)^2}\)
\item predict \(\hat{y}^*\) as the majority vote over \(y^{(1)}, \dots, y^{(k)}\)
\item ties are resolved by picking the outcome class lowest value or alphabetical rank
\end{itemize}
\end{frame}


\begin{frame}{Example: k-nearest-neighbor classification}
\begin{itemize}
\item For illustration we use data from the "Palmer Penguin" data.
\item Features are penguins' bill length (\(x_1\)) and bill depth (\(x_2\)) and outcome (\(y\)) is the penguin's sex (male or female).
\item The kNN classifier (\(k=5\)) separates the \(x_1\)/\(x_2\) plane into regions for "female" and "male", separated by a decision boundary.
\end{itemize}

\begin{center}
\includegraphics[width=.4\textwidth]{fig/knn-data.png}
\includegraphics[width=.4\textwidth]{fig/knn-k5surface.png}
\end{center}
\end{frame}

\begin{frame}{Example: k-nearest-neighbor classification}
\begin{itemize}
\item The parameter \(k\) controls the "roughness" of the classifier. 
\item Larger values of \(k\) produce smoother decision boundaries.
\begin{center}
\includegraphics[width=.3\textwidth]{fig/knn-data.png}\\
\includegraphics[width=.9\textwidth]{fig/knn-kdependence.png}
\end{center}
\item Is \(k\) a trainable parameter that can be selected by empirical loss minimisation?
\end{itemize}
\end{frame}

\begin{frame}{Example: k-nearest-neighbor classification}
\begin{itemize}
\item Split the data randomly into training and test data set:
\begin{itemize}
\item Only the training data set is used by the kNN classifier.
\item Training and test data are used separately to calculate the 0-1 loss (aka accuracy, aka proportion correct) of the classifier.
\end{itemize}
\end{itemize}
\begin{minipage}{.5\textwidth}%
\begin{center}
\includegraphics[width=\textwidth]{fig/knn-cvloss.png}
\end{center}
\end{minipage}%
\begin{minipage}{.5\textwidth}%
\begin{itemize}
\item The training accuracy is perfect for \(k=1\) (Why?) and goes down as \(k\) increases.
\item The test accuracy is best for \(k\approx 6\).
\end{itemize}
\end{minipage}%
\begin{itemize}
\item The neighborhood parameter \(k\) can be optimised by loss minimisation, but should use a different data set than the one used to define the kNN classifier.
\end{itemize}
\end{frame}


\begin{frame}{In-sample vs Out-of-sample error}
\begin{itemize}
\item Our goal is to train a model that generalises well to new data. 
\item Loss on training data (in-sample error) is too optimistic because the model has seen the data.
\item Overfitting: A model achieves very low training loss by "regurtitating" the training data, including any noise and accidental patterns. Overfitted models generalise poorly to new data.
\item Loss on previously unseen data (out-of sample error) is what we are really interested in.
\item It is good practice to remove a fraction of data (e.g. 20\%) during training and use only for model testing.
\end{itemize}
\end{frame}




\begin{frame}{Further reading}
\begin{itemize}
\item Linear algebra for least squares regression.
\item Alternative loss functions
\item Multiple linear regression for multiple inputs.
\item Logistic regression for classification.
\item Radius-neighborhood classifier and other neighborhood methods.
\item Neighborhood methods for missing data imputation.
\item Uncertainty estimation: Resampling methods, bootstrapping.
\end{itemize}
\end{frame}


\section{Tree-based methods}

\subsection{Decision trees}

\begin{frame}{Background}
\begin{itemize}
\item Tree-based methods are simple and powerful function approximation methods.
\item During training, the feature space is split up into rectangular regions.
\item The model's prediction within each rectangular region is a constant.
\end{itemize}
\end{frame}


\begin{frame}{Decision tree example}
\begin{minipage}{.5\textwidth}
\includegraphics[width=\textwidth]{fig/tree-simpleexample.png}
\end{minipage}%
\begin{minipage}{.5\textwidth}
\begin{center}
\includegraphics[width=.8\textwidth]{fig/tree-graph.png}
\end{center}
\end{minipage}%
\end{frame}


\begin{frame}{Decision tree terminology}
\begin{itemize}
\item Split: Partitioning data based on a feature and threshold value
\item Node: a point in the tree where a decision is made
\item Root node: first split in the tree
\item Internal/decision node: Any node that performs a split
\item Leaf node: Final node that outputs a prediction
\item Depth: Number of decision levels from Root node to leaves
\item Impurity: Measure how mixed the classes are
\item Information gain: Reduction in impurity after a split
\end{itemize}
\end{frame}

\begin{frame}[fragile]{Training a decision tree} 
\begin{lstlisting}[language=python]
from sklearn.tree import DecisionTreeClassifier

# initialise decision tree with set maximum depth
clf = DecisionTreeClassifier(max_depth=10)

# X: shape (n_samples, n_features) 
# y: shape (n_samples,) with 0/1 encoded class label

# train decision tree
clf.fit(X, y)
\end{lstlisting}
\end{frame}


\begin{frame}{Decision tree for tree cover classification}
Data set (plot), no code required it's on previous slide, decision surface
\end{frame}


\begin{frame}[fragile]{Visualising trained decision trees}
\begin{minipage}{.4\textwidth}
\begin{lstlisting}[language=python]
from sklearn import tree
clf = DecisionTreeClassifier(
        max_depth=3)
clf.fit(X, y)
plt.figure(figsize=(10, 6))
tree.plot_tree(clf,
  feature_names=["altitude", 
                 "aspect"],
  class_names=["not_fir", 
               "fir"],
  filled=True, 
  rounded=True,
  fontsize=8)
plt.show()
\end{lstlisting}
\end{minipage}%
\begin{minipage}{.6\textwidth}
\includegraphics[width=\textwidth]{fig/tree-vis.png}
\end{minipage}%
\end{frame}


\begin{frame}{How to find node features and thresholds?}
MORE HERE: at each node one of the available features needs to be compared to a threshold - how to select feature and threshold?, impurity measure, information gain, worked example on a decision stump and comparison to code
\end{frame}


\subsection{Random Forest, XGBoost}


\begin{frame}{Random Forest}

\begin{itemize}
\item A RF is an ensemble of decision trees.
\item Each tree is trained on random subsets of training data and input features.
\item Each tree makes a prediction and the forest aggregates them.
\begin{itemize}
\item Classification: Majority vote
\item Regression: Average
\end{itemize}
\item Improves robustness and avoids overfitting compared to single decision tree.
\end{itemize}
\end{frame}



\begin{frame}{Example}

MORE HERE: show code, example data, single tree vs RF decision surfaces

\end{frame}


\begin{frame}{Boosting}
\begin{itemize}
\item In ensemble methods (such as RF), each classifier (tree) is independently trained and optimised for maximum accuracy.
\item A boosting algorithm trains a chain of weak classifiers sequentially. 
\item Core algorithm:
\begin{itemize}
\item train a simple model
\item calculate its errors over the training data
\item train the next model to take the same inputs but predict the previous model's errors
\item repeat, then aggregate all models (often with weights)
\end{itemize}
\item Throughtout the sequence the model is improved by gradually refining the prediction.
\end{itemize}
\end{frame}


\begin{frame}{Gradient Boosting, XGBoost}
\begin{itemize}
\item Gradients of the loss function \(g_i = \frac{\partial L(\hat{y}_i,y_i)}{\partial \hat{y}_i}\) approximate errors.
\item Changing \(\hat{y}_i\) proportional to \(-g_i\) will decrease the loss.
\item In gradient boosting, each tree predicts the negative gradient \(\frac{\partial L(\hat{y},y)}{\partial \hat{y}}\) of the previous tree
\item XGBoost is a popular library for Gradient Boosting
\item MORE HERE
\end{itemize}
\end{frame}



\begin{frame}{OLD Background: Measuring surprise}
\begin{itemize}
\item Shannon and Weaver (1948) developed a mathematical theory of communication
\item As part of this they needed a mathematical measure of "surprise" after observing the outcome of a random variable, which has \(n\) possible outcomes with probabilities \(p_1, \dots, p_n\)
\item Based on simple considerations (additivity, continuity) they showed that if outcome \(k\) is observed, the only reasonable measure of surprise is
\[-\log_2 p_k\]
\item A high-probability outcome is less surprising than a low-probability outcome.
\end{itemize}
\end{frame}


\begin{frame}{OLD Background: Entropy}
\begin{itemize}
\item Entropy is the negative expected surprise
\[H(p_1,\dots,p_n) = \sum_{i=1}^n p_i \log_2 p_i\]
\item Entropy is a summary measure of uncertainty or information content of a probability distribution.
\item The higher entropy, the more "deterministic" a distribution is.
\item TODO: A few bar plots of distributions with their entropy values.
\end{itemize}
\end{frame}

\begin{frame}{OLD Background: Empirical entropy}
\begin{itemize}
\item Given a set of values \(S = (y_1,\dots,y_n)\) where each \(y_i\) is either 0 or 1 
\item From \(S\) we estimate the probability that \(y_i = 1\) by \[\hat{p} = \frac1n \sum_{i=1}^n y_i\] 
\item The probability that \(y_i = 0\) is then estimated by \(1-\hat{p}\).
\item We calculate the empirical binary entropy of the sample \(S\) as the entropy of the distribution \((\hat{p},1-\hat{p})\) \[H(S) = \hat{p} \log_2 \hat{p} + (1-\hat{p})\log_2 (1-\hat{p})\]
\end{itemize}
\end{frame}

\begin{frame}{OLD Decision stump}
\begin{itemize}
\item We have a data set of input/output pairs \(S = \{(x_i,y_i)\}_{i=1}^n\) where 
\item \(x_i \in \mathbb{R}\) are continuous inputs (eg temperature) and 
\item \(y_i \in \{0,1\}\) are binary outputs (eg rain occurrence)
\item We want to find a threshold \(\tau\) to "optimally separate" the data set \(S\) into \(S_l\) and \(S_r\) by \[S_l = \{(x_i, y_i): x_i < \tau\} \quad S_r = \{(x_i,y_i): x_i > \tau\}\]
\item How to choose the threshold? What does "optimally separate" mean here?
\end{itemize}
\end{frame}


\begin{frame}{OLD Decision stump}
\begin{itemize}
\item After setting the threshold \(\tau\) a fraction \(q_l = \frac{|S_l|}{|S|}\) of data ends up left of the threshold and \(q_r = \frac{|S_r|}{|S|}\) on the right
\item Each set has its individual empirical entropy \(H(S_l)\) and \(H(S_r)\)
\item 
\end{itemize}
\end{frame}



\section{Neural Networks}




\end{document}

